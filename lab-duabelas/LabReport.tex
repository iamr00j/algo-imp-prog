\documentclass{article}

\title{COMP26120 Lab 13: Background}
\author{Ru Jie Tham}

\begin{document}
\maketitle

% PART 1 %%%%%%%%%%%%%%%%%%%%%%%%%%%%%%%%%%%%%%%%%%%%%%%%%%%%%%%%%%%%%%%%%%%%%%

\section{The small-world hypothesis}
\label{sec:small world}
% Here give your statement of the small-world hypothesis and how you
% are going to test it.
The small world hypothesis tests the degrees of separation between people. This experiment surfaced from the
examination of the average path length for social networks of people in the US. The idea of "six degrees of 
separation" developed, whereby anyone is connected to anyone through six or fewer steps away. I think the idea
of six degrees is actually shrinking, and the degrees of separation is existing only through people's random 
acquaintaces, and not friendships.

This hypothesis will be tested using the given social network data from Caltech and University of Oklahoma. The
data will be represented as graphs - people as nodes. Random individuals will be picked to find the shortest path
between the chosen individuals. To prove that there are less than six degrees, there should be individuals who are
not connected to one another, or connected by more than five steps away. Graph search functions would be used to
find the shortest path between nodes with heap functions 


\section{Complexity Arguments}
\label{sec:complexity}
% Write down the complexity of Dijkstra's algorithm and of Floyd's algorithm.
% Explain why, for these graphs, Dijkstra's algorithm is more efficient.
Dijkstra's algorithm is a single-source shortest path algorithm. It computes a compact representation of all the
shortest paths from a node to every other node, in the form of a rooted tree. 
Floyd's algorithm is an all-pairs shortest path algorithm - it computes the shortest paths between every pair of
nodes. 
Time complexity of Dijkstra depends on the implementation; it can be O(N^2 + M), O(MlogN), or O(M + NlogN).
Floyd runs in O(N^3) time.
Since we need to find the shortest path from one person to another (not necessarily connected), it is more efficient to implement Dijkstra's
algorithm.

% PART 2 %%%%%%%%%%%%%%%%%%%%%%%%%%%%%%%%%%%%%%%%%%%%%%%%%%%%%%%%%%%%%%%%%%%%%%

\section{Part 2 results}
\label{sec:part2}
% Give the results of part two experiments.
A random number generator is used and the number 17 is used as the starting node for both
social networks to test the small world hypothesis. In Caltech University, an approximate of 20%
of the nodes are connected within a distance of 5 nodes. Most of the nodes are not reachable by
node 17, such as node 14,21, and 28. In this test, it is obeserved that the nodes are either connected
to the starting node within the distance of 5 or less, or they are not at all reachable.
In the case of University of Oklahoma, there are also nodes that are not reachable from the source node 17,
like node 16981 - 16991. Unlike Caltech University, there exist nodes with a distance of 6 from the source 
node 17 in University of Oklahoma, like node XXX from node 17.
From these tests, the social networks proved that the network is not at all small, hence the small world
hypothesis is false. 

% PART 3 %%%%%%%%%%%%%%%%%%%%%%%%%%%%%%%%%%%%%%%%%%%%%%%%%%%%%%%%%%%%%%%%%%%%%%

\section{Part 3 complexity analysis}
\label{sec:complexity3}
% Give the complexity of the heuristic route finder.
The maximum out-degree of the node is chosen as my heuristic method. This method runs in O(N*V), where
N is the number of nodes in the graph and V is the number of out-degrees of the nodes.

\section{Part 3 results}
\label{sec:part3}
% Give the results of part two experiments.
This method is not accurate as compared to Dijkstra's shortest path algorithm. There are some errors,
for instance, the distance of node 21 from node 17 is XX; however in the heuristic method it is said that
the distance between both nodes is 3. It is also observed that the nodes are not reachable by node 17 with
Dijkstra are also not reachable with the heuristic method. That said, some of the nodes that are not reachable
by node 17 using heuristics are reachable with Dijkstra. For instance, the distance of node XXX from node 17 is X
with Dijkstra, but the target node is not reachable with heuristics. Nevetherless, the heuristic method is always
accurate when the distance between the source and the target is 1.


\end{document}
